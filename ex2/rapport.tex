\documentclass[a4paper]{article}\usepackage[]{graphicx}\usepackage[]{color}
%% maxwidth is the original width if it is less than linewidth
%% otherwise use linewidth (to make sure the graphics do not exceed the margin)
\makeatletter
\def\maxwidth{ %
  \ifdim\Gin@nat@width>\linewidth
    \linewidth
  \else
    \Gin@nat@width
  \fi
}
\makeatother

\definecolor{fgcolor}{rgb}{0.345, 0.345, 0.345}
\newcommand{\hlnum}[1]{\textcolor[rgb]{0.686,0.059,0.569}{#1}}%
\newcommand{\hlstr}[1]{\textcolor[rgb]{0.192,0.494,0.8}{#1}}%
\newcommand{\hlcom}[1]{\textcolor[rgb]{0.678,0.584,0.686}{\textit{#1}}}%
\newcommand{\hlopt}[1]{\textcolor[rgb]{0,0,0}{#1}}%
\newcommand{\hlstd}[1]{\textcolor[rgb]{0.345,0.345,0.345}{#1}}%
\newcommand{\hlkwa}[1]{\textcolor[rgb]{0.161,0.373,0.58}{\textbf{#1}}}%
\newcommand{\hlkwb}[1]{\textcolor[rgb]{0.69,0.353,0.396}{#1}}%
\newcommand{\hlkwc}[1]{\textcolor[rgb]{0.333,0.667,0.333}{#1}}%
\newcommand{\hlkwd}[1]{\textcolor[rgb]{0.737,0.353,0.396}{\textbf{#1}}}%

\usepackage{framed}
\makeatletter
\newenvironment{kframe}{%
 \def\at@end@of@kframe{}%
 \ifinner\ifhmode%
  \def\at@end@of@kframe{\end{minipage}}%
  \begin{minipage}{\columnwidth}%
 \fi\fi%
 \def\FrameCommand##1{\hskip\@totalleftmargin \hskip-\fboxsep
 \colorbox{shadecolor}{##1}\hskip-\fboxsep
     % There is no \\@totalrightmargin, so:
     \hskip-\linewidth \hskip-\@totalleftmargin \hskip\columnwidth}%
 \MakeFramed {\advance\hsize-\width
   \@totalleftmargin\z@ \linewidth\hsize
   \@setminipage}}%
 {\par\unskip\endMakeFramed%
 \at@end@of@kframe}
\makeatother

\definecolor{shadecolor}{rgb}{.97, .97, .97}
\definecolor{messagecolor}{rgb}{0, 0, 0}
\definecolor{warningcolor}{rgb}{1, 0, 1}
\definecolor{errorcolor}{rgb}{1, 0, 0}
\newenvironment{knitrout}{}{} % an empty environment to be redefined in TeX

\usepackage{alltt}

\usepackage{fullpage}
\usepackage{amsmath}
\IfFileExists{upquote.sty}{\usepackage{upquote}}{}

\begin{document}

%<<my-label, eval=TRUE, dev='png'>>=
%library(spam) # load the data
%str(Oral) # see structure of data
%#'data.frame': 544 obs. of 3 variables:
%# $ Y : int 18 62 44 12 18 27 20 29 39 21 . . .
%# $ E : num 16.4 45.9 44.7 16.3 26.9 . . .
%# $ SMR: num 1.101 1.351 0.985 0.735 0.668 . . .
%attach(Oral) # allow direct referencing to Y and E
%# generate some plots
%library(fields, warn.conflict=FALSE)
%library(colorspace)
%col <- diverge_hcl(8) # blue - red
%# alternative colors
%# col <- rev(gray(0:8 / 8)) # gray scales
%# col <- rev(heat_hcl(64))
%# use the function provided by spam
%map.landkreis(log(Oral$Y),col=col)
%map.landkreis(Oral$Y/Oral$E,col=col)
%@

\section*{Excercise 1}
\subsection*{a)}
We want to find the full posterior distribution $p\left(\eta, u, \kappa_u, \kappa_v \mid y\right)$, using the following
\begin{align}
\kappa_u & \sim \mathrm{Gamma}(\alpha_u, \beta_u) \\
\kappa_v & \sim \mathrm{Gamma}(\alpha_v, \beta_v) \\
v \mid \kappa_v & \sim \mathcal{N}\left(\mathbf{0}, \kappa_v^{-1}\mathbf{I}\right) \\
p(u \mid \kappa_u) & \propto \kappa_u^{(n-1)/2} \exp\left(-\frac{\kappa_u}{2}u^TRu\right) \\
\eta \mid u, \kappa_v & \sim \mathcal{N}\left(u, \kappa_v^{-1}\mathbf{I}\right) \\
y_i \mid \eta_i & \sim \mathrm{Pois}\left(e_i \exp(\eta_i)\right).
\end{align}
We proceed as follows:
\begin{equation}
\begin{aligned}
p\left(\eta, u, \kappa_u, \kappa_v \mid y\right) & \propto p(y \mid \eta, u, \kappa_u, \kappa_v) \, p(\eta, u, \kappa_u, \kappa_v) \\
& \propto \prod_i p(y_i \mid \eta_i) \, p(\eta \mid u, \kappa_v) \, p(u \mid \kappa_u) \, p(\kappa_u) \, p(\kappa_v).
\end{aligned}
\end{equation}
Writing out the terms we obtain
\begin{equation}
\begin{aligned}
p\left(\eta, u, \kappa_u, \kappa_v \mid y\right)
& \propto \prod_i \frac{e_i^{y_i} \exp(\eta_i y_i) \exp\left(-e_i \exp(\eta_i)\right)}{y_i!} \\
& \cdot \kappa_v^{n/2} \exp\left(-\frac{\kappa_v}{2}\left(\eta - u\right)^T\left(\eta - u\right)\right) \\
& \cdot \kappa_u^{(n-1)/2} \exp\left(-\frac{\kappa_u}{2}u^TRu\right) \\
& \cdot \kappa_u^{\alpha_u - 1} \exp(-\beta_u \kappa_u) \\
& \cdot \kappa_v^{\alpha_v - 1} \exp(-\beta_v \kappa_v).
\end{aligned}
\end{equation}
Multiplying together we obtain
\begin{equation}
\kappa_u^{\frac{n-1}{2} + \alpha_u - 1} \kappa_v^{\frac{n}{2} + \alpha_v - 1}
\exp\left(-\beta_u\kappa_u - \beta_v\kappa_v 
- \frac{\kappa_v}{2}\left(\eta - u\right)^T\left(\eta - u\right)
- \frac{\kappa_u}{2}u^TRu
+ \sum_i \left( \eta_i y_i - e_i \exp(\eta_i)\right)
\right).
\label{eq:full_joint_posterior}
\end{equation}
\subsection*{b)}
Differentiating with respect to $\eta_i$ yields
\begin{align}
f(\eta_i) & = y_i \eta_i - \exp(\eta_i) e_i \\
f'(\eta_i) & = y_i - \exp(\eta_i) e_i \\
f''(\eta_i) & = - \exp(\eta_i) e_i
\end{align}
and inserting this into the Taylor expansion
\begin{equation}
\tilde{f}(\eta_i) = f(\eta_{0_i}) + f'(\eta_{0_i})(\eta_i - \eta_{i_0}) + \frac{1}{2} f''(\eta_{0_i})(\eta_i - \eta_{0_i})^2
\end{equation}
we obtain
\begin{equation}
\begin{aligned}
\tilde{f}(\eta_i) & = \eta_i \left( y_i - \exp(\eta_{0_i}) e_i + \exp(\eta_{0_i}) e_i \eta_{0_i}\right) \\
 & + \eta_i^2 \left( -\frac{1}{2} \exp(\eta_{0_i}) e_i \right) \\
 & - \exp(\eta_{0_i}) e_i + \exp(\eta_{0_i}) e_i \eta_{0_i} - \frac{1}{2} \exp(\eta_{0_i}) e_i \eta_{0_i}^2
\end{aligned}
\end{equation}
which we rewrite as
\begin{equation}
\begin{aligned}
\tilde{f}(\eta_i) &= a_i + b_i \eta_i - \frac{1}{2}c_i \eta_i^2 \\
a_i &= \exp(\eta_{0_i}) e_i ( \eta_{0_i} - \frac{1}{2} \eta_{0_i}^2 - 1) \\
b_i &= y_i + \exp(\eta_{0_i}) e_i (\eta_{0_i} - 1) \\
c_i &= \exp(\eta_{0_i}) e_i.
\end{aligned}
\label{eq:taylor_expansion}
\end{equation}

\subsection*{c (GI)}
The full conditional density $p(\kappa_u \mid y, \kappa_v, \eta, u)$ is proportional to the full joint posterior $p(\eta, u, \kappa_u, \kappa_v \mid y)$ from exercise \emph{a)}. Dropping all terms in \eqref{eq:full_joint_posterior} not depending on $\kappa_u$ we obtain
\begin{equation}
p(\kappa_u \mid y, \kappa_v, \eta, u) \propto
\kappa_u^{\frac{n-1}{2} + \alpha_u - 1}
\exp\left(-\beta_u\kappa_u
- \frac{\kappa_u}{2}u^TRu
\right).
\end{equation}
Doing the same for $p(\kappa_v \mid y, \kappa_u, \eta, u)$ we get
\begin{equation}
p(\kappa_v \mid y, \kappa_u, \eta, u) \propto 
\kappa_v^{\frac{n}{2} + \alpha_v - 1}
\exp\left(- \beta_v\kappa_v 
- \frac{\kappa_v}{2}\left(\eta - u\right)^T\left(\eta - u\right)
\right).
\end{equation}
For $p(u \mid y, \kappa_v, \kappa_u, \eta)$ we obtain
\begin{equation}
\begin{aligned}
p(u \mid y, \kappa_v, \kappa_u, \eta) & \propto
\exp\left(
- \frac{\kappa_v}{2}u^Tu
- \frac{\kappa_u}{2}u^TRu
\right) \\
& \propto
\exp\left(-\frac{1}{2} u^T \left(
\kappa_v \mathbf{I} +
\kappa_u R \right)
u
\right)
\end{aligned}
\end{equation}
which is the well known multivariate normal distribution, with mean zero, and covariance matrix $\Sigma^{-1} = \kappa_v \mathbf{I} + \kappa_u R$.

For $p(\eta \mid y, \kappa_v, \kappa_u, u)$ we get
\begin{equation}
\begin{aligned}
p(\eta \mid y, \kappa_v, \kappa_u, u) & \propto
\exp\left(
- \frac{\kappa_v}{2}\left(\eta - u\right)^T\left(\eta - u\right)
+ \sum_i \left( \eta_i y_i - e_i \exp(\eta_i)\right)
\right) \\
& \propto
\exp\left(
- \frac{\kappa_v}{2}\eta^T\eta + \kappa_v \eta^T u
+ \eta^T y - \exp(\eta)^T e
\right)
\end{aligned}
\label{eq:eta_conditional}
\end{equation}
We use what we got in \eqref{eq:taylor_expansion} to replace the terms $\eta^T y - \exp(\eta)^T e$ in \eqref{eq:eta_conditional}. The $a_i$ is dropped since it does not depend on $\eta$ and we get
\begin{equation}
\begin{aligned}
q(\eta \mid \eta_0, y, u, \kappa_u, \kappa_v)
& \propto \exp\left(-\frac{\kappa_v}{2}\eta^T\eta + \kappa_v \eta^Tu + \eta^T b - \frac{1}{2} \eta^T(\mathbf{I}c)\eta\right) \\
& \propto \exp\left(-\frac{1}{2}\eta^T\left(\mathbf{I}\kappa_v + \mathbf{I}c\right)\eta + \eta^T(b + \kappa_v u)\right)
\end{aligned}
\label{eq:eta_conditional_approx}
\end{equation}

\subsection*{c) (BL)}
The join conditional distribution $p(u, \eta \mid y, \kappa_u, \kappa_v)$ is also proportional to \eqref{eq:full_joint_posterior} and by removing all factors not containing $u$ or $\eta$ we obtain
\begin{equation}
p(u, \eta \mid y, \kappa_u, \kappa_v) \propto \exp\left(-\frac{\kappa_v}{2}(u^Tu - 2\eta^Tu + \eta^T \eta) - \frac{\kappa_u}{2}u^TRu + \eta^T y - \exp(\eta)^T e\right).
\end{equation}
We define $x = (u^T, \eta^T)^T$ and rewrite the previous expression as
\begin{equation}
p(x \mid y, \kappa_u, \kappa_v) \propto \exp\left(\eta^Ty - \exp(\eta)^Te - \frac{1}{2} x^T
\begin{pmatrix}
\kappa_u R + \kappa_v \mathrm{I} & -\kappa_v \mathrm{I} \\
-\kappa_v \mathrm{I}             &  \kappa_v \mathrm{I}
\end{pmatrix}
x
\right)
\label{eq:u_eta_joint}
\end{equation}
Just like in \eqref{eq:eta_conditional_approx} we use what we got in \eqref{eq:taylor_expansion} to replace the terms $\eta^T y - \exp(\eta)^T e$ in \eqref{eq:u_eta_joint}. This gives
\begin{equation}
q(x \mid x_0, y, \kappa_u, \kappa_v) \propto \exp\left(
- \frac{1}{2} x^T
\begin{pmatrix}
\kappa_u R + \kappa_v \mathrm{I} & -\kappa_v \mathrm{I} \\
-\kappa_v \mathrm{I}             &  \kappa_v \mathrm{I}
\end{pmatrix}
x
- \frac{1}{2}\eta^T (\mathrm{I}c) \eta + b^T \eta
\right)
\label{eq:u_eta_joint_approx}
\end{equation}

\end{document}
